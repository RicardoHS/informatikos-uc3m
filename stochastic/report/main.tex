\documentclass[11pt, a4paper]{article}

\usepackage[utf8]{inputenc}
\usepackage[T1]{fontenc}

\usepackage{amsmath,amssymb,amsthm,amsfonts,graphicx,float,fullpage,booktabs, bm} 
\usepackage[colorlinks,citecolor=blue,bookmarks=true]{hyperref}
\hypersetup{linkcolor=blue}
\usepackage{cleveref}
% Code listings
\usepackage{minted}
\usepackage{caption}
\usemintedstyle[python]{manni}
\usemintedstyle[R]{manni}

\usepackage{todonotes}

\title{\vspace{-8ex} \huge \bfseries Stochastic Processes\\
	\LARGE \normalfont \textsc{Assignment One} \vspace{-2ex}}
\author{\bfseries Santiago Alfonso Raposo Briceño - 100414456 \\
		\bfseries Javier Martínez Llamas - 100392684\\
		\bfseries Ricardo Hortelano Sánchez - 100418220}
\date{\vspace{-5ex}} % Remove space where date would be.

\begin{document}
\maketitle
\hrule

\section{Problem 1}

\subsection{Question a}
We can model the different channels as states in a Markov chain, with two absorbing states representing the conversion or non-conversion of a client.
This assumes however that the channel each user explores is only decided by the last channel he or she clicked on, which might not be completely realistic.

\subsection{Question b}
The state space is all of the unique values of the chain existing in our sample. This is easily computed with R, and it appears our state space consists of:
\begin{itemize}
	\item \textbf{Three channels} labeled as \verb|Ch 1|, \verb|Ch 2| and \verb|Ch 3|.
	\item \textbf{Two absorbing states} representing the \verb|Conversion| and \verb|Non-conversion| states.
\end{itemize}

We found the estimate of the transition probabilities using maximum likelihood, as we have from our notes:
\[
	\hat p_{ij} = \frac{n_{ij}}{\sum_{j = 1}^{k}} j,i \in S
\]
Where $n_{ij}$ is the total number of transitions from state $i$ to state $j$ and the denominator is the total amount of transitions \emph{into} state $j$.
The estimate found for our sample is the following transition matrix:
\[
 P = \begin{bmatrix}
 	0.3042 & 0.0792 & 0.1518 & 0.3132 & 0.1515 \\
 	0.1660 & 0.1612 & 0.3365 & 0.1601 & 0.1761 \\
 	0.1358 & 0.0756 & 0.1459 & 0.2202 & 0.4225 \\
 	0.0000 & 0.0000 & 0.0000 & 1.0000 & 0.0000 \\
 	0.0000 & 0.0000 & 0.0000 & 0.0000 & 1.0000
 \end{bmatrix},
\]
where we have encoded our states as numbers using the following mapping: \verb|Ch 1| is row (and column) 1, \verb|Ch 2| is number 2, \verb|Ch 3| is number 3, \verb|Conversion| is number 4 and \verb|Non-Conversion| is number 5.

\subsection{Question c}
Our first consumer journey is \verb|Ch 3 -> Ch 2 -> Conversion|, which is equivalent in our numerical mapping to $X_1 = 3, X2 = 2, X3 = 4$.
The probability we then have to calculate is $P(X_3 = 4|X_2=2, X_1 = 3)$. This can be done as follows:
\begin{align*}
	&P(X_3 = 4|X_2=2, X_1 = 3) = P(X_3 = 4 | X_2 = 2) * P(X_2 = 2 | X_1 = 3) * P(X_1 = 3) \\
	&= P_{2,4} * P_{3,2} * (\bm{\alpha}P)_5
\end{align*}
Assuming our initial distribution $\bm{\alpha} = (\frac{1}{3}, \frac{1}{3}, \frac{1}{3}, 0, 0)$\footnote{We are assuming the client cannot start in one of the conversion or non-conversion states and that it is uniform on the rest, which seems reasonable.} the answer is $0.0026$.

\subsection{Question d}
We can find the limiting distribution for the chain by using the result from equation 3.11 in Dobrow. This allows us to compute the limiting submatrix as $ \bm{(I-Q)^{-1}R}$ for the absorbing states, which is:
\[
	\bm{(I-Q)^{-1}R} = \begin{bmatrix}
		0.5887 & 0.4113 \\
		0.4650 & 0.5350 \\
		0.3925 & 0.6075
	\end{bmatrix}
\]
Where the first column represents the limiting distribution of the conversion state and the second column represents the limiting distribution of the non-conversion state. 
We can then compute the total conversion ratio as the ratio between the sum of column 1 (which is the total proportion of realizations that end in a conversion) divided by the sum of the matrix.
\[
	\text{Conversion Ratio }  = 0.4821.
\]

\subsection{Question e}

\end{document}
