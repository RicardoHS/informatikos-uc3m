\documentclass[11pt, a4paper]{article}
\usepackage[utf8]{inputenc}
\usepackage[T1]{fontenc}

\usepackage{amsmath,amssymb,amsthm,amsfonts,graphicx,float,fullpage,booktabs,bm} 
\usepackage[colorlinks,citecolor=blue,bookmarks=true]{hyperref}
\hypersetup{linkcolor=blue}
\usepackage{cleveref}
% Code listings
\usepackage{minted}
\usepackage{caption}
\usemintedstyle[python]{manni}
\usemintedstyle[R]{manni}

\usepackage{todonotes}

\title{\vspace{-8ex} \huge \bfseries Stochastic Processes\\
	\LARGE \normalfont \textsc{Assignment Two} \vspace{-2ex}}
\author{\bfseries Santiago Alfonso Raposo Briceño - 100414456 \\
	\bfseries Javier Martínez Llamas - 100392684\\
	\bfseries Ricardo Hortelano Sánchez - 100418220}
\date{\vspace{-5ex}} % Remove space where date would be.

\begin{document}
	\maketitle
	\hrule

\section{Exercise 1}

\section{Exercise 2}
\subsection*{Question a}
\todo[inline]{Revisar solucion, no estoy 100\% seguro de que el infinitesimal generator sea tan sencillo de obtener...}
The process $\{\bm{X}_t, \ t\geq 0\}$ is a Homogeneous Poisson process, as we are told that the number of costumers in the system at time $t$ follows a Poisson distribution with rate $\lambda t$. 
We also know that the rate $\lambda$ is constant, so the \emph{increments} are stationary and independent.
It is not specified in the problem, but in order for the process to be a Poisson Process we also require that $X_0 = 0$.

The \emph{state space} of our process is then $\{0\} \cup \mathbb{N}$. The infinitesimal generator of a Poisson process is
\[
Q = \begin{bmatrix}
	-\lambda & \lambda & 0 & 0 & \dots \\
	0 & -\lambda &\lambda  & 0 & \dots \\
	0 & 0 & -\lambda &\lambda & \dots \\
	0 & 0 & 0 & -\lambda & \dots \\
	\vdots &\vdots &\vdots &\vdots & \ddots \\
\end{bmatrix},
\]
as can be seen in example 5.4 of the course slides.

\subsection*{Question b}

\end{document}